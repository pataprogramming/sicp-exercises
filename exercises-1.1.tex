% Created 2014-03-09 Sun 17:41
\documentclass[11pt]{article}
\usepackage[utf8]{inputenc}
\usepackage[T1]{fontenc}
\usepackage{fixltx2e}
\usepackage{graphicx}
\usepackage{longtable}
\usepackage{float}
\usepackage{wrapfig}
\usepackage{rotating}
\usepackage[normalem]{ulem}
\usepackage{amsmath}
\usepackage{textcomp}
\usepackage{marvosym}
\usepackage{wasysym}
\usepackage{amssymb}
\usepackage{hyperref}
\tolerance=1000
\author{Plant Kingdom}
\date{\today}
\title{exercises-1.1}
\hypersetup{
  pdfkeywords={},
  pdfsubject={},
  pdfcreator={Emacs 24.3.50.1 (Org mode 8.2.5h)}}
\begin{document}

\maketitle
\tableofcontents

\section{Section 1.1}
\label{sec-1}
\subsection{Section 1.1.6}
\label{sec-1-1}
\subsubsection{Exercise 1.1: Interpreting expressions}
\label{sec-1-1-1}
\begin{enumerate}
\item Problem
\label{sec-1-1-1-1}
Below is a sequence of expressions.  What is the
result printed by the interpreter in response to each expression?
Assume that the sequence is to be evaluated in the order in which
it is presented.

\begin{verbatim}
10
(+ 5 3 4)
(- 9 1)
(/ 6 2)
(+ (* 2 4) (- 4 6))
(define a 3)
(define b (+ a 1))
(+ a b (* a b))
(= a b)
(if (and (> b a) (< b (* a b)))
    b
    a)
(cond ((= a 4) 6)
      ((= b 4) (+ 6 7 a))
      (else 25))
(+ 2 (if (> b a) b a))
(* (cond ((> a b) a)
         ((< a b) b)
         (else -1))
   (+ a 1))
\end{verbatim}
\item Answer
\label{sec-1-1-1-2}

\begin{verbatim}
> 10
10
> (+ 5 3 4)
12
> (- 9 1)
8
> (/ 6 2)
3
> (+ (* 2 4) (- 4 6))
-16
> (define a 3)
> (define b (+ a 1))
> (+ a b (* a b))
19
> (= a b)
#f
> (if (and (> b a) (< b (* a b)))
      b
      a)
4
> (cond ((= a 4) 5)
        ((= b 4) (+ 6 7 a)))
16
\end{verbatim}
\end{enumerate}
\subsubsection{Exercise 1.2: Translating to prefix notation}
\label{sec-1-1-2}

Translate the following expression into prefix form.

$$\frac{5 + 4 + ( 2 - (3 = (6 + \frac{4}{5})))}{3 (6 - 2)(2 - 7)$$

\begin{verbatim}
(/ (+ 5 4 (- 2 (- 3 (+ 6 (/ 4 5)))))
   (* 3 (- 6 2) (- 2 7)))
\end{verbatim}
\subsubsection{Exercise 1.3: Summing the squares of two larger numbers}
\label{sec-1-1-3}
\begin{enumerate}
\item Problem
\label{sec-1-1-3-1}

Define a procedure that takes three numbers as arguments and
returns the sum of the squares of the two larger numbers.
\item Answer
\label{sec-1-1-3-2}

\begin{verbatim}
(define (square a) (* a a))

(define (sum-of-squares a b) (+ (square a) (square b)))

(define (sum-of-squares-of-larger-two a b c)
  (cond
   ((and (<= a b) (<= a c))  (sum-of-squares b c))
   ((and (<= b a) (<= b c))  (sum-of-squares a c))
   (else                     (sum-of-squares a b))))

(define (sum-of-squares-of-larger-two-alt a b c)
  (if (>= a b)
      (if (>= b c)    ; we know a is in, as it's at least as large
                      ; as one other number
          (sum-of-squares a b)     ; c is out
          (sum-of-squares a c))    ; b is out
      (if (>= a c)    ; we know b is in for this branch, since
                      ;it's strictly larger than a
          (sum-of-squares b a)     ; c is out
          (sum-of-squares b c))))  ; a is out

;; This can be made even shorter by exploiting the return value of if
(define (sum-of-squares-of-larger-two-altalt a b c)
  (if (>= a b)
      (sum-of-squares a (if (>= b c) b c))
      (sum-of-squares b (if (>= a c) a c))))
\end{verbatim}
\end{enumerate}
\subsubsection{Exercise 1.4: Evaluating combinations}
\label{sec-1-1-4}
\begin{enumerate}
\item Problem
\label{sec-1-1-4-1}
Observe that our model of evaluation allows for combinations
whose operators are compound expressions.  Use this observation
to describe the behavior of the following procedure:

\begin{verbatim}
(define (a-plus-abs-b a b)
  ((if (> b 0) + -) a b))
\end{verbatim}
\item Answer
\label{sec-1-1-4-2}

When the expression containing the \texttt{if} operator is evaluated, if \texttt{b}
is strictly larger than 0, the expression returns \texttt{+}, which is used
as the operator for the containing expression, adding \texttt{a} and
\texttt{b}. Otherwise, the \texttt{if} returns \texttt{-}, subtracting \texttt{b} from \texttt{a}; since
\texttt{b} is negative in this case (or 0) this increases the value by the
same amount as it would have increased were \texttt{b} to be positive.
\end{enumerate}
\subsubsection{Exercise 1.5: Applicative order vs normal order}
\label{sec-1-1-5}
\begin{enumerate}
\item Problem
\label{sec-1-1-5-1}
Ben Bitdiddle has invented a test to determine whether the
interpreter he is faced with is using applicative-order
evaluation or normal-order evaluation.  He defines the following
two procedures:

\begin{verbatim}
(define (p) (p))

(define (test x y)
  (if (= x 0)
      0
      y))
\end{verbatim}

Then he evaluates the expression

\begin{verbatim}
(test 0 (p))
\end{verbatim}

What behavior will Ben observe with an interpreter that uses
applicative-order evaluation?  What behavior will he observe with
an interpreter that uses normal-order evaluation?  Explain your
answer.  (Assume that the evaluation rule for the special form
`if' is the same whether the interpreter is using normal or
applicative order: The predicate expression is evaluated first,
and the result determines whether to evaluate the consequent or
the alternative expression.)
\item Answer
\label{sec-1-1-5-2}

The procedure \texttt{p} will cause an infinitely recursive loop if
executed. If Ben is using an applicative-order interpreter, the
operands will be evaluated first, and the command will enter the loop.
If the interpreter is normal order, the value of the operands will not
be resolved immediately. Instead, test will be executed. \texttt{x} will be
resolved in the predicate of the \texttt{if}, which succeeds. As the consequent
branch is followed, \texttt{y} will never be evaluated, and thus \texttt{p} will never
be executed.

To clarify the difference between normal and applicative order,
consider nested applications of the \texttt{sq} procedure:

\begin{verbatim}
(define (sq n) (* n n)
(sq (sq (sq (+ 1 1))))
\end{verbatim}

Suppose this is evaluated in normal order (combining some of the
expansion steps to save space):

\begin{verbatim}
1> (sq (sq (sq (+ 1 1))))
2> (* (sq (sq (+ 1 1)))
      (sq (sq (+ 1 1))))
3> (* (* (sq (+ 1 1))
         (sq (+ 1 1))
      (* (sq (+ 1 1))
         (sq (+ 1 1)))
4> (* (* (* (+ 1 1) (+ 1 1))
         (* (+ 1 1) (+ 1 1)))
      (* (* (+ 1 1) (+ 1 1))
         (* (+ 1 1) (+ 1 1))))
5> (* (* (* 2 2) (* 2 2))
      (* (* 2 2) (* 2 2)))
6> (* (* 4 4) (* 4 4)
7> (* 16 16)
8> 256
\end{verbatim}

As can be seen, none of the "leaf" additions are evaluated until
everything is expanded, so there are eight individual addition and
seven multiplication operations performed (as seen in steps 4 through
7). Compare this to an applicative-order evaluation:

\begin{verbatim}
1> (sq (sq (sq (+ 1 1))))
2> (sq (sq (sq 2)))
3> (sq (sq (* 2 2)))
4> (sq (sq 4))
5> (sq (* 4 4))
6> (sq 16)
7> (* 16 16)
8> 256
\end{verbatim}

Only a single addition and three multiplications need to be performed!
\end{enumerate}
\subsection{Section 1.1.7}
\label{sec-1-2}
\subsubsection{Exercise 1.6: \texttt{if} and the reason for special forms}
\label{sec-1-2-1}
\begin{enumerate}
\item Problem
\label{sec-1-2-1-1}

Alyssa P. Hacker doesn't see why \texttt{if} needs to be
provided as a special form.  "Why can't I just define it as an
ordinary procedure in terms of \texttt{cond?}," she asks.  Alyssa's friend
Eva Lu Ator claims this can indeed be done, and she defines a new
version of \texttt{if}:

\begin{verbatim}
(define (new-if predicate then-clause else-clause)
  (cond (predicate then-clause)
        (else else-clause)))
\end{verbatim}

Eva demonstrates the program for Alyssa:

\begin{verbatim}
(new-if (= 2 3) 0 5)
5

(new-if (= 1 1) 0 5)
0
\end{verbatim}

Delighted, Alyssa uses \texttt{new-if} to rewrite the square-root program:

\begin{verbatim}
(define (sqrt-iter guess x)
  (new-if (good-enough? guess x)
          guess
          (sqrt-iter (improve guess x)
                     x)))
\end{verbatim}

What happens when Alyssa attempts to use this to compute square
roots?  Explain.
\item Answer
\label{sec-1-2-1-2}

In \texttt{sqrt-iter}, before \texttt{new-if} is called, the arguments are evaluated
in applicative order. This includes the recursive call to
itself. Thus, even when the good-enough? termination condition is
satisfied, the recursion will continue, resulting in an infinite
recursion.
\end{enumerate}
\subsubsection{Exercise 1.7: When \texttt{good-enough?} is not good enough}
\label{sec-1-2-2}
\begin{enumerate}
\item Problem
\label{sec-1-2-2-1}

The \texttt{good-enough?} test used in computing square roots will not
be very effective for finding the square roots of very small
numbers.  Also, in real computers, arithmetic operations are
almost always performed with limited precision.  This makes our
test inadequate for very large numbers.  Explain these
statements, with examples showing how the test fails for small
and large numbers.  An alternative strategy for implementing
\texttt{good-enough?} is to watch how \texttt{guess} changes from one iteration
to the next and to stop when the change is a very small fraction
of the guess.  Design a square-root procedure that uses this kind
of end test.  Does this work better for small and large numbers?
\item Answer
\label{sec-1-2-2-2}
\begin{verbatim}
(define (sqrt-iter guess x)
  (if (good-enough? guess x)
      guess
      (sqrt-iter (improve guess x)
                 x)))

(define (improve guess x)
  (average guess (/ x guess)))

(define (average x y)
  (/ (+ x y) 2))

(define (good-enough? guess x)
  (< (abs (- (square guess) x)) 0.001))

(define (sqrt1 x)
  (sqrt-iter 1.0 x))

(define (percent-changed new old)
  (/ (- new old) old))

(define (small-change? new-guess old-guess)
  (< (abs (percent-changed new-guess old-guess)) 0.001))

(define (sqrt-iter-scaled current-guess last-guess x)
  (if (small-change? current-guess last-guess)
      current-guess
      (sqrt-iter-scaled (improve current-guess x) current-guess x)))

(define (sqrt-scaled x)
  (sqrt-iter-scaled 1.0 100.0 x))
\end{verbatim}

Checking the results:

\begin{verbatim}
(sqrt 10000000.0)        ; Racket's built-in
(sqrt1 10000000.0)       ; Version from text
(sqrt-scaled 10000000.0) ; Scale-sensitive version
\end{verbatim}

\begin{verbatim}
> (sqrt 10000000.0)        ; Racket's built-in
3162.2776601683795
> (sqrt1 10000000.0)       ; Version from text
3162.277660168379
> (sqrt-scaled 10000000.0) ; Scale-sensitive version
3162.277666486375
\end{verbatim}

As can be seen the scale sensitive version does not perform as well at
large scales (as the iterative steps are larger, it is somewhat easier
to satisfy the \texttt{small-change?} predicate.

\begin{verbatim}
> (sqrt 0.00000001)
0.0001
> (sqrt1 0.00000001)
0.03125010656242753
> (sqrt-scaled 0.00000001)
0.00010000000000082464
\end{verbatim}

The value of this version becomes evident at very small scales.  The
original version from the text is more easily satisfied in this case,
as the improvement steps are so tiny.
\end{enumerate}
\subsubsection{Exercise 1.8}
\label{sec-1-2-3}
\begin{enumerate}
\item Problem
\label{sec-1-2-3-1}

Newton's method for cube roots is based on the fact that if $y$ is
an approximation to the cube root of $x$, then a better
approximation is given by the value

$$ \frac{x/y^2 + 2y}{3} $$

Use this formula to implement a cube-root procedure analogous to
the square-root procedure.  (In section 1.3.4we will see how to
implement Newton's method in general as an abstraction of these
square-root and cube-root procedures.)
\item Answer
\label{sec-1-2-3-2}

Extending to cube roots is straightforward. Building on the code in the last
exercise:

\begin{verbatim}
(define (cube x)
  (* x x x))

(define (improve-cubic-guess guess x)
  (average guess (/ (+ (/ x (square guess)) (* 2 guess)) 3)))

(define (cbrt-iter-scaled current-guess last-guess x)
  (if (small-change? current-guess last-guess)
      current-guess
      (cbrt-iter-scaled (improve-cubic-guess current-guess x) current-guess x)))

(define (cbrt x)
  (cbrt-iter-scaled 1.0 100.0 x))
\end{verbatim}

This gives results that do indeed look like an approximation to the
cubic root of a number. Comparing against the results of the Racket
\texttt{expt} function:

\begin{verbatim}
> (cbrt 27)
3.0018696341051916
> (expt 27 1/3)
3.0

> (cbrt 1000)
10.009419213708078
> (expt 1000 1/3)
9.999999999999998

> (cbrt 100000000)
464.61792889946435
> (expt 100000000 1/3)
464.15888336127773

> (cbrt 0.0000001)
0.004645991760883334
> (expt 0.0000001 1/3)
0.00464158883361278
\end{verbatim}
\end{enumerate}
% Emacs 24.3.50.1 (Org mode 8.2.5h)
\end{document}
